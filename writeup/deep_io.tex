\subsection{I/O}
\label{sec:deep:io}

\paragraph{}
As we implemented and optimized each MST algorithm we ran valgrind to
profile our code. The program almost always spent more time reading the
input file than running any other single function. It was clear that
optimizing Kruskal's or Prim's would be of little use if we did not
optimize the text parsing. Our original implementation was a simple loop
that used scanf to read in the values. We reimplemented the parser and used
the following series of optimizations
to get a 2.5x speedup in overall running time.
\begin{itemize}
\item
Use mmap to map file directly into th address space and avoid an extra
buffer copy.
\item
Use custom logic to parse the numbers out of the text.
\item
Use fadvise() to give a hint to the kernel that the mmaped file will be
read sequentially. That way the kernel does aggressive read ahead of the
file.
\item
When compiling for part 1, use simpler logic to take advantage of the exactly
one decimal place format of the edge weights.
\end{itemize}
% Briefly discuss how we read in data. Touch on our two approaches as well as
% relevant optimizations (quantify them; a graph may be overkill, but a table or
% at least quoting performance numbers in text would be great).
% \de{need to write this}
