\subsection{Graph Generation}
\label{sec:deep:graphgen}
\paragraph{}
We carefully designed several algorithms to generate graphs.  Each algorithm
generates a particular kind of input, or is better for certain kinds of target
graphs.  In this section, we discuss those algorithms and important aspects of
their creation (like random number generation).

A crucial aspect of input generation is the quality of the random number
generator.  To this end, we used a Mersenne Twister algorithm for generating
random numbers.  Furthermore, we chose the random seed for this algorithm based
on a number generated from a truly random number generator -
\texttt{/dev/urandom}.  \texttt{/dev/urandom} is a Linux utility which ``gathers
environmental noise from device drivers and other sources into an entropy
pool.''  This gives us a high degree of confidence that our generated graphs are
representative of a ``true'' random sampling.

We often want to reuse inputs for various consistency reasons (debugging,
performance testing, correctness checking, etc.).  However, with finite disk
space you can only cache so many graphs - the $8,192$ vertex complete graphs
used $1GB$ each!  Our solution was to build a framework which logged inputs -
from a very small amount of information - edge count, a vertex count, and a
random seed - we could deterministically regenerate any graph.  This turned out
to be quite useful, especially when farming out performance testing operations
to multiple machines.

We spent a significant amount of time carefully studying how graphs could be
efficiently generated.  The running time analysis and empirical studies on which
ones are better in practice for particular edge densities are sadly omitted for
lack of space.  The code is in \texttt{generate\_input.py}.

\begin{description}
  \item \textbf{Random Vertex Positions.} $O(|V|^2)$.  Simply generate
    coordinates for each vertex and then compute and output the pairwise edge
    weights.

  \item \textbf{Complete Graph.} $O(|V|^2)$.  Simply generate uniform random
    edge weights in the desired range for all pairs of vertices.  This is a
    special case because it is easy and efficient to generate a complete graph
    like this.

  \item \textbf{Other, Ensure Connectivity.} $O(|V|)$.  To ensure the graph is
    connected, we iterate over each vertex and connect it to a random vertex in
    the group of vertices we have already iterated over.  For the remainder of
    the edges, choose one of the following two based on the graph's density:


  \item \textbf{Dense.}  $O(V^2 \cdot log(V^2))$.  Enumerate all edges (pairs of
    vertices) and insert them into a heap with a \textit{random key}.  Use
    \texttt{RemoveMin} to remove the desired number of edges and these will be
    the edges in your graph (once an edge is picked a random weight is chosen
    for it).

  \item \textbf{Sparse.}  $O(E \cdot V^2 / (V^2 - E))$.  Use a hash-table to
    keep track of all edges which have been added to the graph.  Pick two random
    vertices; if the edge is not in the graph, add it.  Repeat until the desired
    number of edges have been added to the graph.  This method is very direct
    and has a small amount of work per iteration so in practice it often
    outperforms dense until the graphs become very dense.
\end{description}
